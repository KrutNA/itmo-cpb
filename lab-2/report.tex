% Created 2019-12-09 Пн 03:02
% Intended LaTeX compiler: pdflatex
\documentclass[11pt]{article}
\usepackage[utf8]{inputenc}
\usepackage[T1]{fontenc}
\usepackage{graphicx}
\usepackage{grffile}
\usepackage{longtable}
\usepackage{wrapfig}
\usepackage{rotating}
\usepackage[normalem]{ulem}
\usepackage{amsmath}
\usepackage{textcomp}
\usepackage{amssymb}
\usepackage{capt-of}
\usepackage{hyperref}
\usepackage[T2A]{fontenc}
\usepackage[a4paper,left=3cm,top=2cm,right=1.5cm,bottom=2cm,marginparsep=7pt,marginparwidth=.6in]{geometry}
\usepackage{cmap}
\usepackage{xcolor}
\usepackage{listings}
\usepackage{polyglossia}
\setdefaultlanguage{russian} \setotherlanguage{english}
\setmainfont{Liberation Serif}
\setsansfont{Liberation Sans}
\setmonofont[Contextuals=Alternate,Ligatures={TeX}]{Fira Code Regular}
\usepackage{array}\usepackage{tabularx}
\author{Krutko Nikita / KrutNA}
\date{\today}
\title{}
\hypersetup{
 pdfauthor={Krutko Nikita / KrutNA},
 pdftitle={},
 pdfkeywords={},
 pdfsubject={},
 pdfcreator={Emacs 26.1 (Org mode 9.1.9)}, 
 pdflang={Russian}}
\begin{document}

\newcolumntype{V}[1]{>{\centering}m{\dimexpr #1\textwidth - 2\tabcolsep-1.3333333\arrayrulewidth}}
\large
\thispagestyle{empty}
\begin{center}
\textbf{Национальный Исследовательский Университет ИТМО}\\
\textbf{Факультет Программной Инженерии и Компьютерной Техники}\\
\end{center}
\vspace{2em}
\begin{center}
\includegraphics[width=120pt]{itmo-logo.png}
\end{center}
\LARGE
\vspace{5em}
\begin{center}
\textbf{Вариант №-7}\\
\textbf{Лабораторная работа №2}\\
\Large
\textbf{по дисциплине}\\
\LARGE
\textbf{\emph{'Основы профессиональной деятельности'}}\\
\end{center}
\vspace{11em}
\large
\begin{flushright}
\textbf{Выполнил:}\\
\textbf{Студент группы P3113}\\
\textbf{\emph{Крутько Никита} : 242570}\\
\textbf{Преподаватель:}\\
\textbf{\emph{Перминов Илья Валентинович}}\\
\end{flushright}
\vspace{4em}
\large
\begin{center}
\textbf{Санкт-Петербург 2019 г.}
\end{center}
\pagebreak{}
\setcounter{tocdepth}{2}
\tableofcontents
\vspace{2em}
\normalsize
\section{Задание}
\label{sec:org498addd}
По выданному преподавателем варианту определить функцию, вычисляемую программой, область представления и область допустимых значений исходных данных и результата, выполнить трассировку программы, предложить вариант с меньшим числом команд. При выполнении работы представлять результат и все операнды арифметических операций знаковыми числами, а логических операций набором из шестнадцати логических значений.

\section{Текст исходной программы}
\label{sec:org8cc7921}
\begin{center}
\begin{tabular}{|c|c|c|l|}
\hline
Адрес & Значение & Мнемоника & Комментарий\\
\hline
041 & A04B & LD 04B & Записать в AC значение по адресу O4B (AC = 0200)\\
042 & 0100 & HLT & Отключение ТГ, переход в пультовый режим\\
\hline
043 & A04B & LD 04B & Записать в AC значение по адресу 04B (AC = 0200)\\
044 & 2042 & AND 042 & Выполнить побитовое умножение занчения в AC с ячейкой 042\\
045 & E04D & ST 04D & Записать значение из AC в ячейку 04D  (*04D = 0000)\\
046 & 0200 & CLA & Очистить аккумулятор (AC = 0000)\\
047 & 4041 & ADD 041 & Прибавить к AC значение в ячейке 041 (AC = A04B)\\
048 & 604D & SUB 04D & Вычесть из AC значение в ячейке 04D (AC = A04B)\\
049 & E04C & ST 04C & Записать значение из AC в ячейку 04C (*04C = A04B)\\
04A & 0100 & HLT & Отключение ТГ, переход в пультовый режим\\
04B & 0200 & CLA & Очистить AC\\
04C & A04B & LD 04B & (045: A04B) Записать в AC значение по адресу 04B (AC = 0200)\\
04D & 4041 & NOP & (049: 0000) Не делать ничего\\
\hline
\end{tabular}
\end{center}

\section{Описание программы}
\label{sec:org8b50844}
\subsection{Назначение программы и реализуемая ею функция}
\label{sec:org3aaa963}
Выполенение нижеописанной функции с сохранением промежуточного значения побитовой операции и результата в ячейки 04D и 04C соответсвенно.\\
\(\text{*04C} = \text{*041} - \text{*04D} = \text{*041} - \left(\text{*04B} \mathbin{\land} \text{*042}\right)\)

\subsection{Область представления}
\label{sec:org8e205cb}
*04C, *041, *04B, *04A - знаковые 16-рахрядные числа (1 бит знак, 15 бит - значимые разряды).\\
*04D - 16 бит, результат побитового логического "\emph{AND}".

\subsection{Область допустимых значений исходных данных}
\label{sec:orgb485c8c}
*041, *04B, *042: \(-2^{16-1}\)..\(2^{16-1}-1\)
\(\Rightarrow\) \(-2^{15}\)..\(2^{15}-1\)
\(\Rightarrow\) \(-32768\)..\(-32767\)\\
*04D: \(0\)..\(2^{16}-1\)
\(\Rightarrow\) \(0\)..\(2^{16}-1\)
\(\Rightarrow\) \(0\)..\(65535\)

\subsection{Область допустимых значений результата}
\label{sec:org8bfa7f5}
*04C: \(-2^{16-1}\)..\(2^{16-1}-1\)
\(\Rightarrow\) \(-2^{15}\)..\(2^{15}-1\)
\(\Rightarrow\) \(-32768\)..\(-32767\)

\subsection{Расположение в памяти ЭВМ программы, исходных данных и результатов}
\label{sec:orgadd9fab}
Ячейки 041, 042, 04B: исходные данные.\\
Ячейка 04D - промежуточное значение.\\
Ячейка 04C - результат выполнения.\\
Ячейки 043..04A: операции программы. 

\subsection{Адреса первой и последней выполняемых команд программы}
\label{sec:org6547f23}
043 - адрес точки входа в программу.
04A - адрес пооследней операции программы.
\section{Таблица трассировки}
\label{sec:org74338f0}
\footnotesize
\begin{center}
  \begin{tabularx}{\textwidth}{|V{0.08}|V{0.08}|V{0.07}|V{0.07}|V{0.07}|V{0.07}|V{0.07}|V{0.07}|V{0.07}|V{0.08}|V{0.13}|V{0.13}|}
    \hline
    \multicolumn{2}{|V{0.16}|}{\textbf{Выполненная команда}}
    & \multicolumn{8}{V{0.58}}{\textbf{Содержимое регистров процессора после выполнения команды}}
    & \multicolumn{2}{|V{0.24}|}{\textbf{Ячейка, содежимое которой изменилось после выполнения команды}} \cr \hline
    Адрес& Код & IP  &  CR  & AR  &  DR  & SP  &  BR  &  AC  & NZVC &Адрес& Новый код\cr\hline
    043 & A04B & 044 & A04B & 04B & 0200 & 000 & 0043 & 0200 & 0001 &&\cr\hline
    044 & 2042 & 045 & 2042 & 042 & 0100 & 000 & 0044 & 0000 & 0101 &&\cr\hline
    045 & E04D & 046 & E04D & 04D & 0000 & 000 & 0045 & 0000 & 0101 & 04D & 0000\cr\hline
    046 & 0200 & 047 & 0200 & 046 & 0200 & 000 & 0046 & 0000 & 0101 &&\cr\hline
    047 & 4041 & 048 & 4041 & 041 & A04B & 000 & 0047 & A04B & 1000 &&\cr\hline
    048 & 604D & 049 & 604D & 04D & 0000 & 000 & 0048 & A04B & 1001 &&\cr\hline
    049 & E04C & 04A & E04C & 04C & A04B & 000 & 0049 & A04B & 1001 & 04C & A04D\cr\hline
    04A & 0100 & 04B & 0100 & 04A & 0100 & 000 & 004A & A04B & 1001 &&\cr\hline 
  \end{tabularx}
\end{center}
\normalsize
\section{Вариант программы с меньшим числом команд}
\label{sec:org0f19ad0}
\begin{center}
\begin{tabular}{|c|c|c|l|}
\hline
Адрес & Значение & Мнемоника & Комментарий\\
\hline
041 & A04B & LD 04B & Записать в AC значение по адресу O4B (AC = 0200)\\
042 & 0100 & HLT & Отключение ТГ, переход в пультовый режим\\
\hline
043 & A04B & LD 04B & Записать в AC значение по адресу 04B (AC = 0200)\\
044 & 2042 & AND 042 & Выполнить побитовое умножение занчения в AC с ячейкой 042\\
045 & E04D & ST 04D & Записать значение из AC в ячейку 04D  (*04D = 0000)\\
046 & A041 & LD 041 & Записать в AC значение из ячейки 041 (AC = A04B)\\
047 & 604D & SUB 04D & Вычесть из AC значение в ячейке 04D (AC = A04B)\\
048 & E04C & ST 04C & Записать значение из AC в ячейку 04C (*04C = A04B)\\
049 & 0100 & HLT & Отключение ТГ, переход в пультовый режим\\
04B & 0200 & CLA & Очистить AC\\
04C & A04B & LD 04B & (045: A04B) Записать в AC значение по адресу 04B (AC = 0200)\\
04D & 4041 & NOP & (049: 0000) Не делать ничего\\
\hline
\end{tabular}
\end{center}

\section{Вывод}
\label{sec:org7d63bec}
В ходе лабораторной работы я познакомился с внутренним устройстом БЭВМ-NG, её командами, научился вводить в неё данные и выполнять программы, выполнять трассировку, аналищировать по результатам этого сокращать программы.
\end{document}